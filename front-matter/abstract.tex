\chapter*{Abstract}\label{ch:abstract}
 $\beta$-decay studies of nuclei in the neutron-rich region and far from the valley of stability are vital to understanding nuclear structure evolution and r-process nucleosynthesis. $\beta$-decay strength functions, delayed neutron emission probabilities ($P\textsubscript{n,2n}$) and energy spectra of delayed neutrons are some of the main indicators of the structural complexities of these nuclei, and act as valuable information for simulating r-process pathways. A few facilities around the world are capable of synthesizing nuclei in the \textit{terra incognita} region of the nuclei chart and lying on the r-process pathway. These fragmentation facilities are capable of providing a number of isotopes to study at the same time, requiring a proper identification system to tag these nuclei with atomic mass (Z) and mass-to-charge (A/Q) ratio using heavy-magnet and energy-degrader settings. For the sake of decay studies, a segmented scintillator YSO (Y\textsubscript{2}SiO\textsubscript{5}:Ce doped) based implantation detector was developed at the University of Tennessee, Knoxville. The detector is compact in structure and offers good spatial and timing resolution, crucial for ion-$\beta$ correlations and time-of-flight (ToF) based $\beta$-delayed neutron emission spectroscopy, respectively. The detector was employed as a part of the BRIKEN neutron counter at Radioactive Ion Beam Factory (RIBF) at RIKEN Nishina Center in Japan, aiming to measure $P_{n,2n}$ for nuclei around the {\textsuperscript{78}Ni} region. Another variant of the detector having a more advanced design was used along with VANDLE (Versatile Array for Neutron Detection at Low Energy) to conduct spectroscopy of $\beta$-delayed neutrons in the same region. In this experiment, a direct measurement of energy spectra will provide information about Gamow-Teller strength distributions. It will also directly verify the conclusions from the BRIKEN experiment where evidence for dominating single-neutron emission from two-neutron unbound states is seen in \textsuperscript{84-87}Ga isotopes.